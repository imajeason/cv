%!TEX TS-program = xelatex
\documentclass[]{friggeri-cv}

\begin{document}
\header{潘}{兴华}{}


% In the aside, each new line forces a line break
\begin{aside}
  \section{关于我}
    12/1983, 河北清河
    +86 151 1024 5471
    ~
    \href{mailto:scupython@gmail.com}{scupython@gmail.com}
    \href{http://panxhorn.me}{http://panxhorn.me}
    \href{https://github.com/scupython/}{github://scupython}
  \section{Languages}
    English (CET-6)
  \section{Programming}
    {\color{red} $\varheartsuit$} Python (flask)
    JavaScript (Emberjs)
    C/C++ 
    CSS \& HTML
\end{aside}
\section{Tar 求职目标}
1. 云计算, Big Data或Openstack\\
2. Linux C/C++ 开发

\section{Exp 工作经历}
\begin{entrylist}
  \entry
    {\small 11/11 至今}
    {系统工程师}
    {智明星通/行云, 北京}
    {Web运维平台}
    {采用Python Flask微框架独立重写Web管理平台,主要实现功能包括:项目管理,主机信息管理,主机监控管理,成本统计,主机信息和成本的Excel导出,主机初始化部分功能。此外使用Flask-Login和Flask-Principal提供系统登陆和权限配置。\\
      系统组件选择包括:\\
      Web后端: Nginx, Uwsgi, Python in virtualenv, Mysql, SQLAlchemy\\
      Web前端:Bootstrap, JQuery, EmberJS。}
    \entry{}{}{}
     {Openstack}
      {搭建OpenStack管理虚拟机,制作Glance镜像包括Centos, Ubuntu, Windows Server 2008, 采用Horizon Web管理界面, Volume提供额外存储。虚拟机主要提供VPN服务,内部测试机,VPS替代,线上Windows Server等。}      
  \entry
    {\small 08/10-08/11}
    {实习}
    {中国石油大学, 北京}
    {油藏管理\\系统开发}
    {Java (Spring + Hibernate), Flex 3}
\end{entrylist}
\section{Edu 教育背景}
{\small
\begin{tabularx}{\textwidth}{llR}
  09/09-07/12&\hspace{1.4cm}软件工程硕士&中国石油大学 信息学院 \\
  09/03-07/07&\hspace{1.4cm}应用化学学士&四川大学 化学学院
\end{tabularx}
}
\section{Skp 技能点}
\begin{entrylist}
  \entry
    {}{}{}{\emph{Linux}}
    {\normalfont{熟练Linux操作,熟悉Shell编程, 掌握Sed, Awk。了解内核配置及常用优化。熟悉嵌入式Linux开发,交叉编译环境。}}
  \entry
    {}{}{}{\emph{Python}}
    {\normalfont{Flask微框架,Flask-Admin, Flask-login, Flask-Principal, SQLAlchemy, Jinja2。}}
  \entry
    {}{}{}{\emph{English}}
    {\normalfont{CET-6,熟练阅读专业文献资料,基本英文写作。}}
  \entry
    {}{}{}{\emph{其他}}
    {\normalfont{JavaScript, CSS/HTML, C/C++, Java, ActionScript, Qt, MySQL\\ \LaTeX, PHP, Git, JSON, Ajax, OpenStack, Ganglia, Nagios, Ansible.}}
\end{entrylist}

\section{Ent 兴趣爱好}
  Linux Kernel, LLVM, Android, FP, Big Data\\
  魔兽世界\\
  Google Plus, Douban(FM), 163. 
\end{document}
